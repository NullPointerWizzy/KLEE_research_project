\documentclass{article}
\usepackage{hyperref}
\hypersetup{
    colorlinks,
    citecolor=black,
    filecolor=black,
    linkcolor=black,
    urlcolor=black
}

\title{Rapport : Projet de Sudoku}
\author{Vieira Do Vale Thomas}
\date{\today}

\begin{document}
\maketitle

\newpage

\renewcommand*\contentsname{Sommaire}
\tableofcontents

\newpage
\section{Impélémentation de Gird Parser} 
Lors du deuxième TD, nous avions pour tâche d'implémenter un  
\textit{grid\_parser}. Le principe de se parser consiste à prendre une grille 
de sudoku stockée dans un fichier et à la stocker dans une structure de données 
appelée "grid". Dans un premier temps, j'ai mis en œuvre les fonctions
 utilitaires demandées, ce qui s'est avéré relativement simple. En revanche, 
 l'implémentation du parser à poser des défis plus complexes.
\\ \\
La principale difficulté réside dans la gestion des nombreux cas différents 
qui peuvent se présenter. Il est impératif de transformer correctement la grille
standard de base, mais aussi de gérer les grilles comportant des espaces 
superflus, des lignes vides, des commentaires, et même les cas où la grille 
fournie n'est pas valide. En effet, si le nombre de colonnes ou de lignes est 
incorrect, ou s'il y a un caractère incorrect, une alerte doit être générée. 
\\ \\
Ma stratégie pour implémenter le parser a consisté à commencer par créer une 
fonction capable de parser correctement une grille normale. Ensuite, j'ai 
récupéré les différentes grilles que nous devions être en mesure de parser. J'ai
soumis chaque grille à mon parser, puis j'ai corrigé les problèmes qui se \\ 
présentaient. Cependant, cette approche a abouti à un code qui réussissait tous
les tests, mais qui manquait de lisibilité, car il n'était pas structuré de 
manière claire. 
\\ \\
Face à cette problématique, j'ai décidé de recommencer le travail à partir de 
zéro. Pour cela, j'ai pris l'initiative d'imprimer le code afin de l'étudier et
de déterminer quelles parties pouvaient être regroupées de manière plus 
cohérente. J'ai également opéré des choix de conception différents. Dans ma 
première version, j'utilisais une première boucle pour récupérer la première 
ligne, puis je passais à une double boucle for pour gérer la taille des lignes
et des colonnes. Dans ma nouvelle implémentation, j'ai opté pour une seule
boucle while en fonction du caractère récupéré, me permettant ainsi d'utiliser
des instructions switch case. Cette modification rend plus clair le traitement
de chaque cas.
\\ \\
Un autre changement majeur visant à améliorer la lisibilité de mon code a été 
d'abandonner la gestion d'erreur à chaque cas. Auparavant, je libérais la 
mémoire, définissais une erreur à true, et fermait le fichier à chaque cas de 
sortie. Pour éliminer cette répétition de code, j'ai eu recours à des 
instructions "goto".


\newpage
\section{Implémentation de Lone Number} 
Dans cette section, nous aborderons l'implémentation de l'heuristique du "lone 
number". Le principe de cette heuristique est de déterminer si une couleur n'est
présente que dans une seule cellule. Pour ce faire, j'ai commencé par mettre en
place une implémentation naïve de cet algorithme. La méthode consiste en une 
boucle qui parcourt tous les indices. À l'intérieur de cette boucle, nous 
examinons l'ensemble de la sous-grille pour vérifier si l'indice apparaît plus 
d'une fois. Si c'est le cas, nous ne faisons rien ; sinon, nous parcourons à 
nouveau la sous-grille pour identifier l'unique cellule contenant l'indice, 
puis nous supprimons les autres couleurs présentes dans cette cellule. Cette 
approche a une complexité de \(O(n^2)\), avec $n$ représentant la taille de 
l'entrée.
\\ \\
Il m'est apparu évident qu'il existait des possibilités d'amélioration 
significative de cet algorithme. Ma première idée a été de stocker la première 
occurrence de chaque indice, afin d'éviter de refaire un deuxième parcours de la
sous-grille. Cependant, cette amélioration n'a pas modifié la complexité. Il 
devenait donc impératif de trouver une solution bien plus efficace. Après 
réflexion, il est apparu possible de réaliser l'algorithme en \(O(n)\), 
toujours avec $n$ comme taille de l'entrée. Pour ce faire, il était nécessaire 
de repérer tous les "lone numbers" en un seul passage, puis de les supprimer 
dans un deuxième passage, sans que ces deux passages soient imbriqués.
\\ \\
L'idée pour repérer les "lone numbers" consiste à créer deux sous-ensembles : 
le premier, que nous appellerons judicieusement \textit{first\_appearance}, 
pour stocker les éléments lors de leur première apparition, et le deuxième, 
appelé \\ \textit{second\_appearance}, pour stocker les éléments lors de leur 
deuxième apparition. Pendant notre premier passage, nous remplissons ces 
sous-ensembles, mais cette fois-ci en examinant les cellules plutôt que les 
indices.
\\ \\
Pour remplir \textit{first\_appearance}, nous effectuons l'union de 
\textit{first\_appearance} avec la sous-grille examinée. En procédant ainsi, 
nous nous assurons de rassembler tous les éléments déjà vus une fois. Pour 
\textit{second\_appearance}, nous devons identifier les éléments présents à la 
fois dans \textit{first\_appearance} et dans la sous-grille examinée, ce qui 
nécessite une intersection entre les deux ensembles. Il est crucial de réaliser 
cette opération avant l'union, sinon nous considérerions que tous les éléments 
de la sous-grille ont déjà été vus deux fois, ce qui peut ne pas être le cas.
\\ \\
Une fois les deux sous-ensembles remplis, les "lone numbers" sont les éléments 
présents dans \textit{first\_appearance} mais absents de 
\textit{second\_appearance}. Nous les identifions donc en effectuant 
l'intersection entre les deux ensembles. La dernière étape consiste à trouver 
les cellules contenant ces "lone numbers". Nous parcourons la grille en faisant
l'intersection entre la cellule examinée et l'ensemble des "lone numbers". Si 
cette intersection n'est pas triviale, nous remplaçons la cellule par cette 
intersection, qui représente notre "lone number". Il est essentiel de ne pas 
effectuer ce remplacement si l'intersection est vide ou si la cellule examinée 
est un singleton, car cela évite de créer des changements en boucle, et donc 
que le programme ne se termine jamais.


\section{Implémentation de Hidden Subset}
Dans cette section, nous explorerons l'implémentation de l'heuristique du \\ 
"hidden-subset". Le principe fondamental de cette heuristique consiste à \\ 
rechercher un sous-ensemble de taille $n$ présent dans $n$ cellules. Il est 
important de noter que ce sous-ensemble n'est pas nécessairement complet dans 
toutes les cellules. Lorsqu'un tel sous-ensemble est identifié, il devient 
possible de supprimer toutes les autres couleurs présentes dans les cellules 
où l'intersection avec l'ensemble n'est pas vide.
\\ \\
L'implémentation de cette heuristique s'est avérée être la plus complexe parmi 
toutes. J'ai fait une première version qui respectait strictement l'heuristique.
Mais le code de cette heuristique n'était absolument pas performant. Lors de 
différents tests, je me suis rendu compte que si je restreignais les 
sous-ensembles cachés à partir d'une cellule, je pouvais améliorer grandement 
l'efficacité. Le sous-ensemble recherché est donc contenu obligatoirement dans 
une cellule, mais n'est pas obligé d'être la cellule entière.
\\ \\
Mon nouvel algorithme fonctionne de la manière suivante : pour chaque cellule, 
deux opérations distinctes sont effectuées. Premièrement, nous comptons le 
nombre de cellules ayant une intersection non vide avec la cellule en cours 
d'examen, en stockant dans une liste les cellules qui satisfont cette condition. 
\\ \\
La deuxième opération consiste à vérifier si le nombre de cellules trouvées est
équivalent au nombre de couleurs de la cellule en cours d'examen. Dans ce cas,
nous sommes autorisés à supprimer les éléments qui ne font pas partie de 
l'intersection de ces cellules avec la cellule en cours. La conservation des 
cellules dans ce scénario particulier permet d'éviter de parcourir l'ensemble 
complet de la sous-grille. Cette approche optimise le processus en se 
concentrant uniquement sur les cellules pertinentes, améliorant ainsi 
l'efficacité globale de l'algorithme.
\\ \\


\section{Algorithme de génération de sudoku}
Nous allons maintenant explorer l'implémentation du générateur de sudoku. Ma 
stratégie d'implémentation a consisté d'abord à créer un générateur fonctionne
sans le mode unique, puis à ajouter le mode unique par la suite. Il existe 
plusieurs approches pour générer une grille. On peut partir d'une grille vide 
et faire des choix aléatoires pour la remplir en s'arrêtant à notre ratio de 
remplissage, dans notre cas, il est demandé de remplir à hauteur de $75\%$. Une
autre possibilité est de résoudre une grille vide, puis de retirer des cases. 
J'ai préféré opter pour la deuxième option, car il m'a semblé plus simple 
d'implémenter l'unicité dans ce cas.
\\ \\
Maintenant que j'ai défini ma conception, il fallait trouver comment générer une
grille remplie de manière aléatoire. Le moyen le plus simple était de 
réutiliser la fonction de backtrack que nous avons codée pour le solveur. 
Pour ma première version, j'ai envisagé de partir d'une grille vide et 
d'utiliser le backtrack pour la résoudre. Cependant, le problème était que 
cette fonction faisait des choix qui n'étaient pas aléatoires. Il fallait 
donc trouver un moyen d'ajouter cette composante sans altérer le fonctionnement
du backtrack. J'ai donc créé une fonction \textit{choice\_random} qui choisit
au hasard une case qui n'est pas un singleton, puis fait un choix aléatoire
pour la valeur de la case. Cette implémentation pose des problèmes de 
complexité. Il prend beaucoup trop de temps pour générer des grilles de 
taille 64, donc il est nécessaire d'améliorer la vitesse du backtrack et de 
lui fournir une grille plus facile à résoudre.
\\ \\
Pour améliorer le backtrack, la meilleure approche consiste à faire comme dans 
\textit{grid\_choice}, mais en effectuant un choix aléatoire. Le moyen le plus 
simple est de simplement modifier la fonction \textit{grid\_choice} en 
remplaçant \textit{colors\_rightmost} par \textit{colors\_random} au moment du
 choix. Il peut sembler étonnant que cela ne modifie pas les performances du 
 solveur. En réalité, il est possible que cela améliore ou ralentisse le 
 \textit{mode\_first}, mais nous ne pouvions pas garantir que le choix le plus 
 à droite était optimal. Ainsi, prendre une option au hasard ne nous fait rien 
 perdre. Pour le \textit{mode\_all}, on explore toutes les possibilités de 
 toute façon, donc prendre le premier choix de manière aléatoire ne change rien.
\\ \\
Pour donner à la grille plus de contraintes, il est nécessaire de faire des 
choix avant de le soumettre au backtrack. La méthode la plus simple est de 
résoudre un bloc, une ligne et une colonne. Ainsi, on est sûr d'avoir toujours 
une grille cohérente, mais on réduit considérablement les possibilités du 
backtrack. Pour effectuer ces choix, j'ai décidé de commencer par remplir le 
premier bloc (celui en $(0,0)$), puis de remplir les lignes et blocs en 
appliquant les heuristiques entre chaque choix. Les choix sont évidemment faits
de manière aléatoire.
\\ \\
Malgré tout cela, il arrive des moments où le backtrack ne trouve tout 
simplement pas de solution en un temps raisonnable. La solution trouvée est de 
lancer le backtrack, mais s'il dépasse un certain temps, alors on abandonne et 
on relance. Pour implémenter cette méthode, on utilise le temps, stocké lors du 
lancement de la fonction. À chaque récursion du backtrack, on vérifie si on a 
dépassé le temps défini. Cependant, cela ne doit se produire que lorsqu'on le 
lance depuis le générateur. Pour cela, j'ai ajouté un mode appelé 
\textit{mode\_generate}. Pour déterminer le temps optimal, des tests ont été 
réalisés avec différents intervalles. Après mes essais, j'ai trouvé que trente 
secondes étaient le temps idéal.
\\ \\
Maintenant, avec une grille résolue en main, il faut supprimer un certain nombre
de cases. La première version utilisait simplement une boucle \textit{for} qui
retirait un élément calculé aléatoirement. Pour déterminer la case à retirer, 
on imagine la grille comme une seule grande ligne. On choisit donc un chiffre 
entre $0$ et $size*size$, puis on calcule à quelle case cela correspond dans 
la grille. Une fois cela fait, nous avons le générateur fonctionnant pour le 
mode unique. Pour passer en mode unique, il suffit de modifier la deuxième 
partie. La solution que j'ai choisie pour faire fonctionner le mode unique 
est de vérifier à chaque fois que je retire un élément que la solution est 
unique. Si ce n'est pas le cas, je rétablis le choix. Je pourrais utiliser le 
\textit{mode\_all}, mais cela prendrait du temps pour rien parfois. Je veux 
m'arrêter dès que j'ai trouvé deux solutions. J'ai donc ajouté un nouveau mode,
le \textit{mode\_unique}. Avec ce nouveau mode, il suffit de lancer le backtrack 
avec cette option, puis s'il trouve deux solutions, alors je rétablis le choix.
\\ \\
Cet algorithme a des performances acceptables, bien que variables. Il se peut 
qu'il trouve une solution dès la première tentative de génération, mais il se 
peut aussi qu'il ne trouve jamais de solution, bien que cela soit peu probable.
Malgré cela, le temps moyen d'exécution du programme reste acceptable.



\section{Amélioration du code}

Le code pouvait être amélioré à plusieurs niveaux, tant en termes de qualité que
de performance.
\\ \\
En ce qui concerne la qualité, j'ai utilisé l'outil Valgrind pour vérifier 
l'absence de fuites de mémoire dans divers scénarios. Lorsque des fuites de 
mémoire étaient détectées, j'analysais l'origine de l'erreur et tentais de 
comprendre le moment opportun pour libérer la mémoire. Malheureusement, cette 
tâche nécessitait souvent plusieurs tentatives, en particulier dans la phase de 
backtrack. En effet, lors du backtrack, une copie de la grille était transmise,
et le backtrack renvoyait également une grille, qui pouvait être identique à 
celle fournie ou créée par le backtrack. Il était donc essentiel de faire 
preuve d'une grande prudence à ce niveau.
\\ \\
Pour optimiser les performances, j'ai consacré du temps à améliorer \\ 
l'algorithme des différentes heuristiques et fonctions. Deux grandes parties 
ont été examinées : les heuristiques et la génération. Pour les heuristiques, 
j'ai lancé le solveur sur toutes les grilles disponibles dans le dossier 
\textit{grid\_solver}. Pour pouvoir vérifier qu'une amélioration était 
fonctionnelle, il fallait que je vérifie deux choses, que je trouve toujours le
bon nombre de solutions, et que j'étais plus rapide qu'avant. Pour me 
simplifier la tâche, j'ai écrit deux scripts bash, et vérifiant que je trouve 
les mêmes nombres de solutions que votre programme, et l'autre qui lance les 
tests sur toutes les grilles. Si je trouvais le bon nombre de solutions, et 
que le temps était plus rapide sur toutes grilles, alors les améliorations 
était validée. Mais si elle n'améliorait pas le temps sur toutes les grilles 
et même ralentissait sur certains, il fallait comparer le temps gagner et 
perdu. Un cas particulier était celui des heuristiques "hidden subset". Il 
était possible de définir une condition sur la taille des ensembles à examiner,
mais il était crucial de déterminer le moment optimal pour effectuer cette 
vérification. Pour ce faire, j'ai réalisé des tests empiriques en suivant la 
même méthodologie que précédemment. Les résultats ont montré qu'aucune 
amélioration significative n'était obtenue en ne considérant pas toutes les
possibilités.
\\ \\
Pour augmenter la vitesse du générateur, le paramètre principal était le temps 
accordé au backtrack pour générer une grille. Étant donné que le temps de
génération d'une grille était aléatoire, la comparaison entre deux exécutions
n'était pas suffisante, nécessitant une moyenne sur dix exécutions pour 
évaluer les différentes options. J'ai constaté que la meilleure durée pour 
mon programme était de vingt secondes.
\end{document}
